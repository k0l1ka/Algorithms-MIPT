\documentclass[a4paper,12pt]{article} % тип документа

% report, book

%  Русский язык

\usepackage[T2A]{fontenc}			% кодировка
\usepackage[utf8]{inputenc}			% кодировка исходного текста
\usepackage[english,russian]{babel}	% локализация и переносы


% Математика
\usepackage{amsmath,amsfonts,amssymb,amsthm,mathtools} 


\usepackage{wasysym}
% мои добьавки-----------------------------------------
\usepackage{ dsfont }
\usepackage{listings}

%мое----------------------------------------------------
%Заговолок
\author{Попов Николай}
\title{Домашнее задание 3 \\ по алгоритмам \LaTeX{}}



\begin{document} % начало документа

\maketitle 
\newpage 

\section*{№1}
Равные числа будут равны НОД-у всех чисел, записанныз на доске, т.к. вычитанием из большего числа меньшее мы реализуем алгоритм евклида ( вычитание - его более медленный вариант, но в результате тоже остается отсаток от деления).

\section*{№2}
Запишем определения для НОК(n, m) = (n, m) и НОД(n, m) = [n, m], зная разложение этих чисел на простые множители (здесь показатели могут быть нулевыми, но основания степеней - простые числа - одни и те же)
 \[n = p_{1}^{a_{1}}*p_{2}^{a_{2}}*...*p_{k}^{a_{k}}\]
\[m = p_{1}^{b_{1}}*p_{2}^{b_{2}}*...*p_{k}^{b_{k}}\]
Получим
\[(n, m)=
\prod\limits_{i}p_{i}^{min(a_{i},b_{i})}\]
\[[n, m]=
\prod\limits_{i}p_{i}^{max(a_{i},b_{i})}\]
Отсюда следует
\[(n, m)*[n, m]= nm\]
т.к. min(a,b)+max(a,b)=a+b\\

Значит, для нахождения НОК надо посчитать произведение чисел, найти их НОД по алгоритму Евклида и разделить первое на второе. \\

Корректность алгоритма следует из выше приведенных выкладок.\\

Пусть n - длина двоичной записи чисел.
Асимптотика складывается из последовательно выполяемых алгоритмов умножения ($\Theta(n^{2})$), алгоритма Евклида($\Theta(n^{3})$) алгоритма деления($\Theta(n^{2})$).В итоге, получаем $\Theta(n^{3})$. \\

(Действителльно, деление и умножение можно сделать например по алгоритмам Divide и Multiply с указанной асимптотикой. В алгоритме Евклида через каждые два рекурентных вызова длина обоих чисел уменьшается вдвое (т.к. если $a \geqslant b$, то a mod b < a/2), а значит, вызовов будет не больше 2n. При этом на каждом шаге выполняется деление, т.е. поярдка $n^{2}$ операций. В итоге, Евклид дает $\Theta(n^{3})$ асимптотику.)

\section*{№3}
Посмотрим, что нужно сделать чтобы посчитать искомую сумму попарно различных (по индексам) элементов массива  длины n. Для простоты сделаем это для 6 чисел massiv[ ] =\{a b c d e f\}:
\[\sum=ab+ac+ad+ae+af+bc+bd+be+bf+cd+ce+cf+de+df+ef=\]\[=a(b+c+d+e+f)+b(c+d+e+f)+c(d+e+f)+d(e+f)+ef\]
Т.е. надо для каждого числа посчитать сумму чисел, стоящих после него, тогда сумма произведений каждого из чисел массива на соответствующую ему частичную сумму и будет искомым результатом. Корректность очевидна.\\
\[
\begin{tabular}{c | c c c c c c|}
sum[]& &b+c+d+e+f&c+d+e+f&d+e+f&e+f&f\\
a[]&a&b&c&d&e&f\\
\end{tabular}\\
\]
\hspace{5cm}Псевдокод\\
res - результат, sum[i+1] - массив частичных сумм элемента a[i], i - вспомогательная переменная.
\begin{lstlisting}
res = 0
sum[n] = a[n]
for i = n-1 to 2 do
	sum[i] = sum[i+1] + a[i]
for i = 1 to n-1 do
	res +=a[i]*sum[i+1]
return res
\end{lstlisting}
Не принимая во внимание время необходимое для совершения ариметических операций, получаем что алгоритм выполнит порядка n-3 шага, а значит является линейным.

\section*{№4}

а)$T(n)= 36 T(\dfrac{n}{6})+n^{2}, a=36,  b=6, f(n)= n^{2}$
\[n^{\log_{b}{a}}=n^{\log_{6}{36}}=n^{2}=\Theta(n^{2})=f(n)=n^{2}\]
Значит, по второму пункту мастер-теоремы, получаем \[T(n)=\Theta(f(n)\log_{}{n}=	\Theta(n^{2}\log_{}{n})\]

б)$T(n)=3T(\dfrac{n}{3})+n^{2}, a=3,  b=3, f(n)= n^{2}$
\[n^{\log_{b}{a}}=n^{\log_{3}{3}}=n\]
\[f(n) = \Theta(n^{2})\]
Проверим регулярность f(n):\\
\[af(n/b)=3f(n/3)=3\dfrac{n^{2}}{9}=\dfrac{1}{3}n^{2}<1*n^{2}=1*f(n)\]
По третьему пункту мастер-теоремы получаем:\\
\[T(n)=\Theta(n^{2})\]

в)$T(n)=4T(\dfrac{n}{2})+\dfrac{n}{\log_{}{n}}$\\
\[n^{\log_{b}{a}}=n^{\log_{2}{4}}=n^{2}\]
Т.к. $f(n)=O(n)$, то по 1 пункту мастер-теоремы получаем $T(n)=\Theta(n^{2})$

\section*{№5}
$T(n)= n *T(\dfrac{n}{2})+O(n)$\\
Расписав дерево рекурсии получаем формулу для числа операций на к-ом шаге:\\
\[\dfrac{n^{k}}{2^{\frac{1}{2}k(k-1)}}T\left(\dfrac{n}{2^{k}}\right)+O \left(\dfrac{n^{k}}{2^{\frac{1}{2}k(k-1)}}\right)\]
Тогда Т(n)=C и при k, меняющемся от 1 до $\log_{2}{n}$, получим:\\
\[\Omega(n^{\log_{2}{\sqrt{2n}}}) = T(n)=Cn^{\log_{2}{\sqrt{2n}}}+\sum\limits_{k=1}^{\log_{2}{n}}O \left(\dfrac{n^{k}}{2^{\frac{1}{2}k(k-1)}}\right)= O \left(n^{\log_{2}{\sqrt{2n}}}\log_{2}{n}\right)\]

\section*{№6}
а)\[
T(n)=T(\alpha n)+T((1-\alpha)n)+Cn
=T(\alpha^{2}n)+2T(\alpha(1-\alpha)n)+T((1-\alpha
)^{2})+2Cn=...\]\[...=kCn + \sum\limits_{i=0}^{k}C_{k}^{i}T(\alpha^{k-i}(1-\alpha)^{i}n)
\]
Оценим по уровням  обрезованного и дополненного деревьев, получим:\\
\[Cn\log_{\frac{1}{\alpha}{n}}\leqslant T(n) \leqslant Cn\log_{\frac{1}{1-\alpha}{n}} \rightarrow T(n) =\Theta(n\log_{}{n})
\]
б)\[T(n)=T(\dfrac{n}{2})+2T(\dfrac{n}{4})+Cn=T(\dfrac{n}{2^{2}})+4T(\dfrac{n}{2*4})+4T(\dfrac{n}{4^{2}})+2Cn=...\]
\[...=
\sum\limits_{i=0}^{k}C_{k}^{i}2^{i}T(\dfrac{n}{2^{k+i}})+kCn\]
Оценка такая же по 2 деревьям:\\
\[Cn\log_{4}{n}\leqslant T(n) \leqslant Cn\log_{2}{n} \rightarrow T(n)=\Theta(n\log_{}{n})\]
в)\[T(n)=27T(\dfrac{n}{3})+\dfrac{n^{3}}{\log^{2}_{}{n}}= 27^{2}T(\dfrac{n}{3^{2}})+\dfrac{n^{3}}{\log^{2}_{}{\dfrac{n}{3}}}=...=27^{k}T(\dfrac{n}{3^{k}})+\sum\limits_{i=1}^{k}\dfrac{n^{3}}{\log^{2}_{}{\dfrac{n}{3^{i-1}}}}=\]
\[=C_{1}n^{3}+n^{3}\sum\limits_{k=1}^{\log_{3}{n}}\dfrac{1}{\log^{2}_{}{\dfrac{n}{3^{k-1}}}}=C_{1}n^{3}+n^{3}\sum\limits_{k=1}^{\log_{3}{n}}\dfrac{1}{(\log_{3}{n}-k+1)^{2}}=\]
Т.к. сумма наша сумма обратных квадратов больше единицы (1 содержится в сумме) и меньше бесконечной суммы  обратных квадратов, равной $\dfrac{\pi^{2}}{6}$, то $T(n) = \Theta(n^{3})$\\
\section*{№7}
Для решения воспользуемся теоремой Вильсона:\\
Натуральное число p>1 - простое, iff (p-1)!+1 кратно p. \\
Тогда
\[- (p-1)!mod p =1 mod p = i! \cdot (i!)^{-1} mod p \rightarrow [ i! \cdot (i!)^{-1}+(p-1)!]\vdots p \rightarrow i![(i!)^{-1}+(i+1)(i+2)...(p-1)]\vdots p \]т.к. i$\leqslant n<p$\[\rightarrow (i!)^{-1} mod p = -(i+1)(i+2)...(p-1)\]
Значит, для нахождения искомого обратного остатка надо для каждого числа i от 1 до n в массиве найти произведение последующих чисел до (p-1)(это делается за линейное время наподобие номера 3, т.е. асимптотика $O(n)$). Остаток от деления этого числа, взятого с минусом, как показано выше и будет искомым результатом.
\end{document}